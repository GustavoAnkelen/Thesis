\documentclass[10pt, twoside,letterpaper]{phstylee}
\usepackage[ansinew]{inputenc}
\usepackage[spanish]{babel}



\setlength{\parindent}{0mm}
\usepackage{amsthm}
\usepackage{mathrsfs} 
\usepackage{color}
\usepackage{mathtools}
\newcommand{\apendice}[1]{\textcolor{cyan}{[#1]}}  
\newcommand{\corregir}[1]{\textcolor{red}{[#1]}}

\usepackage{calrsfs}


\newtheorem{teo}{Teorema}[section]
\newtheorem{cor}{Corolario}[section]
\newtheorem{coro}{Corolario}[section]
\newtheorem{lem}{Lema}[section]
\newtheorem{conj}{Conjetura}[section]
\newtheorem{prop}{Proposici\'on}[section]
\newtheorem{rem}{Observaci\'on}[section]
\newtheorem{obs}{Observaci\'on}[section]
\newtheorem*{defn}{Definici\'on}
\newtheorem{ex}{Ejemplo}[section]
\newtheorem{ejem}{Ejemplo}[section]
\newtheorem{ej}{Ejercicios}[section]
\newtheorem{rec}{Recordatorio}[section]
\newtheorem{notacion}{Notaci\'on}[section]


\newcommand{\s}{\vspace{0.3cm}}
\newcommand{\comp}{{\mathbb{C}}}
\newcommand{\real}{{\mathbb{R}}}
\newcommand{\N}{{\mathbb{N}}}
\newcommand{\Z}{{\mathbb{Z}}}
\newcommand{\Q}{{\mathbb{Q}}}
\newcommand{\R}{{\mathbb{R}}}
\newcommand{\C}{{\mathbb{C}}}
\newcommand{\p}{{\mathbb{P}}}
\newcommand{\im}{\;\operatorname{Im}\;}
\newcommand{\grad}{\operatorname{grad}}
\newcommand{\esf}{{\mathbb{S}}}
\def\w #1{{\widetilde {#1}}}
\def\Id{\operatorname{Id}}
\newcommand{\Bl}{\operatorname{Bl}}

\newcommand{\gr}{{gr}} \newcommand{\Hom}{{Hom}}
\newcommand{\End}{{End}} \newcommand{\Ima}{{Im}}
\newcommand{\Nm}{{Nm}}
\newcommand{\Ric}{\operatorname{Ric}}
\newcommand{\Vol}{\operatorname{Vol}}
\newcommand{\dist}{\operatorname{dist}}
\input Configuracion.tex

%%%%%%%%%%%%% PAQUETE DE ESTILO PROPIO %%%%%%%%%%%%%%%

\usepackage{myStyle}

%%%%%%%%%%%%%%%%%%%%%%%%%%%%%%%%%%%%%%%%%%%%%%%%%%%%%

\usepackage{graphicx}
\usepackage[spanish]{varioref}
\usepackage[all,cmtip]{xy}
\usepackage{amssymb}
\usepackage{xspace}
\usepackage[bookmarks=true,bookmarksnumbered,colorlinks=true,linkcolor=black,citecolor=black,pdfstartview=FitH,linktocpage]{hyperref}
%\usepackage{tocbibind}
\hypersetup{
    colorlinks=true,       % false: boxed links; true: colored links
    linkcolor=blue,          % color of internal links
    citecolor=blue,        % color of links to bibliography
    filecolor=blue,      % color of file links
    urlcolor=blue           % color of external links
}
\usepackage{cancel}

%%% Paquetes de dibujo %%%
\usepackage{tikz-cd}
\usepackage{tikz}

\usepackage{asymptote}

\usetikzlibrary{patterns}
\usepackage{pstricks}
\usepackage{pgfplots}
\usepgfplotslibrary{patchplots}

%%%%	%%%%	%%%%	%%%%

\linespread{1.3} \makeindex \spanishdecimal{.}

%%%%%%%%%%%% ARREGLAR LOS ESPACIOS "JUSTIFICADOS" VERTICALMENTE

\raggedbottom   

\begin{document}

\renewcommand{\listfigurename}{\'INDICE DE FIGURAS}
\renewcommand{\listtablename}{\'INDICE DE TABLAS}
\newcommand{\dd}{\textup{d}}
\newcommand{\id}{\operatorname{Id}}
\newcommand{\cohom}{\operatorname{H}}
\newcommand{\imagen}{\operatorname{im}}
\newcommand{\christoffel}[3]{\ensuremath{\Gamma^{#1}_{#2#3}}}
\newcommand{\RicciCurv}[4]{\ensuremath{R^{#1}_{#2 #3 #4}}}
\newcommand{\distribution}{\mathfrak{D}}
\newcommand{\ii}{{\bf{i}}}
\newcommand{\loc}{\operatorname{loc}}
\newcommand{\sobolev}{\mathcal{H}}
\newcommand{\supp}{\operatorname{supp}}
\newcommand{\tr}{\operatorname{tr}}
% PORTADA DE LA MEMORIA -------------------------

\setcounter{tocdepth}{2}

\input Portada.tex

\cleardoublepage

% RESUMEN DE LA MEMORIA -------------------------

\pagestyle{fancyplain} \pagenumbering{roman}
 
\onehalfspacing
\chapter*{Agradecimientos}
\input Agradecimientos.tex
%
\chapter*{Abstract}
\input Abstract.tex


\chapter*{Resumen}
\input Resumen.tex

\cleardoublepage

% �NDICE E INTRODUCCI�N -------------------------

\cleardoublepage

\setcounter{tocdepth}{1}
\tableofcontents

%\listoffigures

%\listoftables

\cleardoublepage


\pagenumbering{arabic}

\chapter*{Introducci\'on}

\input Introduccion.tex

% CAP�TULOS DE LA MEMORIA ------------------------

\cleardoublepage

\chapter{Variedades de K\"ahler y m\'etricas de Einstein}\label{cap:cap1}

\input cap1.tex

%%%%%%%%%%%%%%%%%%%%%%%%%%%%%%%%%%%%%%%%%%%%%%%%%%%%%%%%%%%%

\cleardoublepage

\chapter{Espacios de H{\"o}lder y Teor\'ia de Schauder}\label{cap:cap2}

\input cap2.tex

%%%%%%%%%%%%%%%%%%%%%%%%%%%%%%%%%%%%%%%%%%%%%%%%%%%%%%%%%%%%

\cleardoublepage

\chapter{Soluciones de Yau y Aubin a la Conjetura de Calabi}\label{cap:cap3}

\input cap3.tex

%%%%%%%%%%%%%%%%%%%%%%%%%%%%%%%%%%%%%%%%%%%%%%%%%%%%%%%%%%%%

\cleardoublepage

\chapter{Consecuencias del Teorema de Yau}\label{cap:cap4}

\input cap4.tex

%%%%%%%%%%%%%%%%%%%%%%%%%%%%%%%%%%%%%%%%%%%%%%%%%%%%%%%%%%%%



%%%%%%%%%%%%%%%%%%%%%%%%%%%%%%%%%%%%%%%%%%%%%%%%%%%%%%%%%%%%

% AP�NDICES DE LA MEMORIA --------------------------
%\cleardoublepage

%\appendix
%\chapter{Variedades Anal\'iticas Complejas y Variedades Algebraicas Complejas}\label{ape:VAR}
%\input apendiceA.tex
%\chapter{Haces y Cohomolog\'ia de \v{C}ech}\label{ape:HC}
%\input apendiceB.tex

% GLOSARIO DE LA MEMORIA ---------------------------

%\cleardoublepage
%\chapter*{Notaci�n y S�mbolos}
%\input glosario.tex

\cleardoublepage \printindex

\cleardoublepage \cleardoublepage

% BIBLIOGRAF�A -------------------------------------

%\chapter*{Bibliograf�a}

%%% IMPORTANTE: Usar mathscinet-ams-org.usm.idm.oclc.org/ para entrar a MathSciNet. Buscar referencia y en la pesta�a "Select alternative format" escoger "BibTeX". Copiar y pegar en archivo .bib. 

%\bibliographystyle{unsrt}   
\bibliographystyle{alpha}
\bibliography{bibliografia}
\end{document}
