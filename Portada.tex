% PORTADA

\pagestyle{empty}

\begin{center}
\begin{figure}[h]\label{USM}
\begin{center}
\includegraphics[scale=0.60]{USM.png}
\end{center}
\end{figure}

\large \textbf{UNIVERSIDAD T�CNICA FEDERICO SANTA MAR�A}

\vspace{1.5mm}

\normalsize DEPARTAMENTO DE MATEM�TICA

\vspace{15mm}

\Large {\bf M�tricas de Einstein y \\
Geometr�a de la Ecuaci�n de Monge-Amp\`ere Compleja}

\vspace{15mm}

\normalsize Memoria de T�tulo presentada por:

\vspace{2.5mm}

\large \textbf{Gustavo Arcaya Espinosa}

\vspace{10mm}

\normalsize como requisito parcial para optar al t�tulo de

\vspace{2.5mm}

\textbf{Ingeniero Civil Matem�tico}

\vspace{2.5mm}

\normalsize Otorgado por la Universidad T�cnica Federico Santa Mar�a

\vspace{5mm}

Profesor Gu�a

%\vspace{2mm}

Dr. Pedro Montero Silva

\vspace{17.5mm}

Junio 2022.


\end{center}

\cleardoublepage

\vspace{5mm}

\noindent T�TULO DE LA MEMORIA:

\vspace{5mm}

\noindent{\large {\bf M�TRICAS DE EINSTEIN Y GEOMETR�A DE LA ECUACION MONGE-AMP\`ERE COMPLEJA\\}}

\vspace{25mm}

\noindent AUTOR:

\vspace{5mm}

\noindent{\large {\bf GUSTAVO ARCAYA ESPINOSA}}

\vspace{20mm}

\noindent TRABAJO DE MEMORIA, presentado en cumplimiento parcial de
los requisitos para el t�tulo de
Ingeniero Civil Matem�tico de la Universidad
T�cnica Federico Santa Mar�a.

\vspace{20mm}

\begin{tabular}{p{50mm}c}
Dr. Salom\'on Alarc\'on & \rule{60mm}{1pt} \\
& \\
& \\
Dr. Pedro Montero & \rule{60mm}{1pt} \\
& \\
& \\
Dr. Giancarlo Urz\'ua & \rule{60mm}{1pt} \\
& \\
& \\
Dr. Alexander Quaas & \rule{60mm}{1pt} \\
&
\end{tabular}

\vspace{10.5mm}

\hfill Valpara�so, Chile, Junio 2022.

\cleardoublepage

\vspace{100mm}

\begin{flushright}



  \textbf{Para Valentina, Alejandra, Cecilia y Valeria} \\


  


\end{flushright}
